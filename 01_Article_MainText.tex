% Please give the surname of the lead author for the running footer
\leadauthor{Tosheva \& Laine}

% info about NM-BC format: https://www.nature.com/nmeth/about/content

\title{NanoJ: a high-performance open-source super-resolution microscopy toolbox for ImageJ}
%NM-BC: The title is limited to 10 words (or 90 characters)
\shorttitle{NanoJ}

% Use letters for affiliations, numbers to show equal authorship (if applicable) and to indicate the corresponding author
\author[1,2\space *]{Kalina L. Tosheva}
\author[1-3\space *]{Romain F. Laine}
\author[1,2,4]{Nils Gustafsson}
\author[1,2,4]{Robert D. M. Gray}
\author[1,2]{Pedro Almada}
\author[1]{?Jason Mercer?}
\author[5]{Christophe Leterrier}
\author[1-3\space\Letter]{Pedro M. Pereira}
\author[1-3\space\Letter]{Si\^{a}n Culley}
\author[1-3\space\Letter]{Ricardo Henriques}

\affil[1]{MRC-Laboratory for Molecular Cell Biology. University College London, London, UK}
\affil[2]{Department of Cell and Developmental Biology, University College London, London, UK}
\affil[3]{The Francis Crick Institute, London, UK}
\affil[4]{Centre for Mathematics and Physics in Life Sciences and Experimental Biology (CoMPLEX), University College London, London, UK}
\affil[5]{Aix Marseille Université, CNRS, INP UMR7051, Marseille, France}
%\affil[6]{London Centre for Nanotechnology, London, UK}
%\affil[7]{Institute for the Physics of Living Systems, University College London, London, UK}

\affil[*]{Equal contributing authors}

\maketitle

%TC:break Abstract
\begin{abstract}
%Lets try to keep the abstract between 70-150 words, I have noticed no guidance
With the advancement of digital microscopy, researchers now have access to a wealth of computational approaches for image data acquisition and analysis. This has allowed findings based on molecular quantification,  dynamics, and structure modelling to become central to fluorescence microscopy. In the field of super-resolution microscopy in particular, image processing tools are indispensable due to the nature of recording of raw data. However, researchers are now confronted with choosing between the numerous available packages for image processing. Over the recent years, our lab has built a self-contained framework of image analysis software with a focus on live-cell super-resolution microscopy, which we called NanoJ, a reference to the popular ImageJ software it was developed for. Here, we highlight the capabilities of using NanoJ at every step of microscopy studies from the image acquisition (NanoJ-Fluidics), artefact correction of the raw data (drift correction, multi-channel alignment) to super-resolution image reconstruction (SRRF), image quality assessment (SQUIRREL) and modelling (NanoJ-VirusMapper) in a single well-established framework.

\end {abstract}
%TC:break main
%the command above serves to have a word count for the abstract

\begin{keywords}
    ImageJ | Fiji | Super-Resolution Microscopy | Image Analysis
\end{keywords}

\begin{corrauthor}
    p.pereira\at ucl.ac.uk, s.culley\at ucl.ac.uk, r.henriques\at ucl.ac.uk
\end{corrauthor}

%%%%%%%%%%%%%%%%
% Introduction %
%%%%%%%%%%%%%%%%
% ------------------------------------------------------------------------------------------------------------------------------------

\subsection*{Introduction}
% RL and RH 500 words max
  Fluorescence microscopy has become ubiquitously used in biological studies since its conception in the 20\textsuperscript{th} century. The advancement of the technique now allows researchers to quantify the dynamic behaviour of specifically labelled molecules and observe the structures they form in living cells. In recent years, super-resolution microscopy has further expanded the toolbox of light microscopy techniques by increasing the detection precision of existing methods down to the single molecule level. However, extracting biologically relevant quantitative information from any type of fluorescence microscopy raw data requires specialised post-processing software.
  \newline
  Here, we present NanoJ, a highly versatile set of digital tools developed to aid in multiple steps of image acquisition and analysis, with particular focus on the demands of live-cell super-resolution imaging. It is available as a series of ImageJ-based plugins which can be used individually or concomitantly to obtain high-quality, artefact-free images and extract quantitative and qualitative information from them. While developed for specific biological questions, NanoJ is aimed at broad applications, thus it is compatible with data acquired with a multitude of microscope setups and experimental workflows. It is open-source, user-friendly and individual modules have written manuals making it an accessible tool for both non-expert users as well as developers.
  NanoJ comprises of general tools, which include drift correction and channel realignment, and specialised algorithms for multi-modular image acquisition and advanced processing. In the following sections, we will briefly describe each of these modalities and demonstrate their application with relevant example data.
  
\subsection*{Sample manipulation while imaging}
% this section is about pumpy (PMP, ask Christophe if he has another 5 colour pumpy dataset), 500 words max
\Blindtext

\subsection*{Image data preparation}
% this section is about drift-correction and channel alignment (?David channel realignment? images?, drift-correction?), 500 words max
\Blindtext

\subsection*{Super-Resolution analysis and rendering}
% this section is about SRRF, Romain, 500 words max
\Blindtext

\subsection*{Super-Resolution quality control}
% this section is about SQUIRREL-ErrorMapping and FRC, Sian, 500 words max
\Blindtext

\subsection*{Super-Resolution data analysis}
% this section is about VirusMapper, Rob + images of archaea and viruses?, 500 words max
Further to providing information on the quality of SR images, NanoJ includes the analytical tool for extracting quantitative structural informationm,called VirusMapper  \cite{gray2016virus,gray2017}. It is the first freely available algorithm which employs high-throughput single particle analysis (SPA) to SR data. Multiple copies of a specific structure can be imaged in a single field of view using SR microscopy and fed into the program to extract an accurate structural model. Thereby, VirusMapper can automatically detect, segment, align, classify and average out thousands of individual instances of this structure based on cross correlation similarity. Notably, the precision of the method of tens of nanometers is sufficient to distinguish nanoscopic changes between separate populations of the same structure under different biological conditions, as was previously demonstrated for the vaccinia protein L4, a constituent of the viral inner core, in free and membrane-bound virus particles \cite{gray2016virus}. For obtaining of more complicated models, VirusMapper can process multicolour SR images, enabling alignment between different structures in the same complex with a precision down to a few nanometers. We illustrate this with models obtained from (STED/SIM) images of three proteins in three distinct parts of vaccinia virus - … (Fig a - Rob). 
\newline
Although it was initially developed for the study of viruses, VirusMapper’s generalised SPA approach and unbiased analysis makes it applicable to any organism, provided a sufficient conservation between copies of the detected structure exist. Furthermore, raw data can be obtained by using a variety of labelling strategies and imaging modalities, making the algorithm broadly compatible with any type of SR method. To demonstrate both of these properties, we additionally modelled the structure of the division ring of the archea Sulfolobus, obtained from (…) images (Fig b - Gabriel). 

(Conclusion and how it ties in with rest of NanoJ)

\subsection*{Discussion and Future Perspectives}
% KT 500 words max
\Blindtext

% ------------------------------------------------------------------------------------------------------------------------------------
\subsection*{Software and Hardware Availability}
NanoJ-Fluidics follows open-source software and hardware standards, it is part of the NanoJ project \cite{gustafsson2016fast, culley2018quantitative,gray2016virus}. The steps to assemble a complete functioning system are described in \href{https://github.com/HenriquesLab/NanoJ-Fluidics/wiki}{https://github.com/HenriquesLab/NanoJ-Fluidics/wiki}. 

\begin{acknowledgements}
    We thank Prof. Ralf Jungmann at Max Planck Institute of Biochemistry Munich for reagents and advice. This work was funded by grants from the UK Biotechnology and Biological Sciences Research Council (BB/M022374/1; BB/P027431/1; BB/R000697/1) (R.H., P.M.P. and R.F.L.), the UK Medical Research Council (MR/K015826/1) (R.H.), the Wellcome Trust (203276/Z/16/Z) (S.C. and R.H.) and the Centre National de la Recherche Scientifique (CNRS ATIP-AVENIR program AO2016) (C.L.). P.A. was supported by a PhD fellowship from the UK’s Biotechnology and Biological Sciences Research Council. C.L.D. was supported by PhD funding from the Medical Research Council, UK (1214605). Research by B.B. was supported by UCL, Cancer Research UK (C1529/A17343), and MRC (MC\_CF12266).K.L.T. is supported
    by a 4-year MRC Research Studentship.
\end{acknowledgements}


\begin{contributions}
    P.A. and R.H. devised the hardware and wrote the software. P.A., P.M.P., C.L. and R.H. planned experiments. Experimental data sets were acquired by P.M.P. (Fig. \ref{fig:LiveToFix}), G.C, F.B.R. and C.L. (Fig. \ref{fig:PAINT}), P.A. and C.L.D. (Fig. \ref{fig:Dix}). Data was analysed by P.A., P.M.P. and S.C. while G.C., B.B., C.L. and R.H. provided research advice. The paper was written by P.A. P.M.P., R.F.L., C.L. and R.H. with editing contributions of all the authors.
\end{contributions}

\begin{interests}
    The authors declare no competing financial interests.
\end{interests}

\section*{Bibliography}
\bibliographystyle{zHenriquesLab-StyleBib}
\bibliography{06_Bibliography_Clean}
