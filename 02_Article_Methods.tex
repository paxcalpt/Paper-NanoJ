\section*{Methods}

\subsection*{Cell lines}
COS7 cells were cultured in phenol-red free DMEM (Gibco) supplemented with 2 mM GlutaMAX (Gibco), 50 U/ml penicillin, 50 \textmu{}g/ml streptomycin (Penstrep, Gibco) and 10\% fetal bovine serum (FBS; Gibco). hTERT-RPE1 cells stably expressing Zyxin-GFP~\cite{dix2018role} were cultured in DMEM F-12 Glutamax (Gibco), with 10\% FBS, 3.4\% sodium bicarbonate (Gibco), 1\% Penstrep. All cells were grown at 37\degree C in a 5\% \ce{CO2} humidified incubator. Cell lines have not been authenticated.

\subsection*{Plasmids}
The plasmid expressing the calponin homology domain of utrophin fused to GFP (GFP-UtrCH) was a gift from William Bement~\cite{burkel2007versatile} (\href{https://www.addgene.org/26737/}{Addgene plasmid \#26737}). 

\subsection*{Antibody conjugation}
Secondary antibodies were labelled with DNA strands (see bellow for details) as previously described~\cite{schnitzbauer2017super}. In short, secondary antibodies were concentrated via amicon 100 kDa spin filters to 2-6 mg/ml. 50-100 \textmu{}g of antibody was labelled using a Maleimide-Peg2-succinimidyl ester (stocks of 10 mg/ml in DMF) for 90 min at 40x molar excess at 4\degree C on a shaker. After the 90 min incubation, unreacted crosslinker was removed via a zeba spin column. Thiolated DNA was reduced using DTT for 2h at room temperature. DTT was separated from the reduced DNA via a Nap5 column and fractions containing DNA were concentrated via 3 kDa amicon spin filters. The reduced DNA was then added to the antibody bearing a functional maleimide group in 25x molar excess and incubated over night at 4\degree C on a shaker in the dark. Antibody-DNA constructs were finally purified via 100 kDa amicon spin filters. DNA-PAINT labelling: \\
- For Mitochondria: Goat anti-Mouse (A28174, ThermoFisher) with I1 (docking: 5’-TTATACATCTA-3'; imager: 5’-CTAGATGTAT-ATTO655-3’); \\
- For Vimentin: Goat anti-chicken (Abcam, ab7113) with I2 (docking: 5’-TTAATTGAGTA-3'; imager: 5’-GTACTCAATT-Cy3B-3’); \\
- For Clathrin: Goat anti-Rabbit (A27033, ThermoFisher). with I3 (docking: 5’-TTTCTTCATTA-3'; imager: 5’-GTAATGAAGA-Cy3B-3’); \\
- For alpha-tubulin: Donkey anti-Rat (A18747, ThermoFisher) with I4 (docking: 5’-TTTATTAAGCT-3'; imager: 5’-CAGCTTAATA-ATTO655-3’). \\
Sequences for DNA-PAINT strands were obtained from~\cite{jungmann2014multiplexed}. Both thiolated and fluorophore conjugated DNA strands were obtained from Metabion. \\

\subsection*{NanoJ-Fluidics framework}
We provide the detailed instructions to easily build and use the system in a regular biology lab, as well as the software enabling its control and automation (Sup. Note \ref{supnote:designandassembly}).

\subsection*{Live-to-Fixed Super-Resolution imaging}
The NanoJ-Fluidics syringe pump array was installed on a Nikon N-STORM microscope equipped with 405, 488, 561 and 647 nm lasers (20, 50, 50 and 125 mW at the optical fiber output). One individual syringe pump module containing the fixative was kept within the incubator of the microscope at 37\degree C. All steps after cell transfection were performed on the microscope, using NanoJ-Fluidics. COS7 cells (kind gift from M.Marsh) were seeded on ultraclean~\cite{pereira2015high} 25 mm diameter thickness 1.5H coverslips (Marienfeld) at a density of 0.3–0.9x10$^{5}$ cells/cm$^{2}$. One day after splitting, cells were transfected with a plasmid encoding  the calponin homology domain of utrophin fused to GFP (GFP-UtrCH) using Lipofectamin 2000 (Thermo Fisher Scientific) according to the manufacturer’s recommendations. Cells were imaged 1-2 days post transfection in culture medium using an Attofluor cell chamber (Thermofisher), covered with the lid of a 35 mm dish (Thermofisher), that was kept in place using black non-reflective aluminum tape (T205-1.0 AT205, THORLABs).\\ 
Cells were fixed at 37\degree C for 15 minutes with 4\% paraformaldehyde in cytoskeleton-preserving buffer (1X PEM, 80 mM PIPES pH 6.8, 5 mM EGTA, 2 mM MgCl2) \cite{leyton2016pfa}. After fixation cells were permeabilised (1X PEM with 0.25\% Triton-X) for 20 min, blocked with blocking buffer (5\% Bovine Serum Albumin (BSA) in 1X PEM) for 30 minutes, and stained with Phalloidin-AF647 (Molecular Probes, 4 units/mL) for 30 minutes.\\
Laser-illumination Highly Inclined and Laminated Optical sheet (HILO) imaging of Utrophin-GFP in live COS7 cells was performed at 37 \degree C and 5\% \ce{CO2} on a Nikon N-STORM microscope. A 100x TIRF objective (Plan-APOCHROMAT 100x/1.49 Oil, Nikon) with additional 1.5x magnification was used to collect fluorescence onto an EMCCD camera (iXon Ultra 897, Andor), yielding a pixel size of 107 nm. For timelapse imaging, 100 raw frames (33 ms exposure) were acquired once every 10 minutes (with the illumination shutter closed between acquisitions) for 150 minutes with 488 nm laser illumination at 4\% of maximum output. STORM HILO imaging of Alexa Fluor 647-phalloidin in fixed cells was performed on the same system. 50,000 frames were acquired with 33 ms exposure and 642 nm laser illumination at maximum output power with 405 nm pumping when required. STORM imaging was performed in GLOX buffer (150mM Tris, pH 8, 1\% glycerol, 1\% glucose, 10mM NaCl, 1\% β-mercaptoethanol, 0.5 mg/ml glucose oxidase, 40 \textmu{}g/ml catalase) supplemented with Phalloidin-Alexa Fluor 647 (1 U/mL).\\

\subsection*{Multiplexed Super-Resolution microscopy with Exchange-PAINT and STORM}
COS-7 cells (obtained from ATCC) were seeded on 18mm, 1.5H glass coverslips (Menzel-Gläser). 24 hours after seeding, they were fixed using 4\% PFA, 4\% sucrose in PEM buffer at 37\degree C \cite{leyton2016pfa}. After blocking in phosphate buffer with 0.022\% gelatin, 0.1\% Triton-X100 for 1.5 hours, cells were incubated with primary antibodies overnight at 4\degree C: mouse monoclonal anti-TOM20 (BD Bioscience \# 612278), rabbit polyclonal anti-clathrin heavy chain (abcam ab21679), chicken polyclonal anti-vimentin (BioLegend \# 919101) and rat anti-alpha-tubulin (mix of clone YL1/2 abcam \# 6160 and clone YOL1/34 Millipore CBL270). After rinses, they were incubated with Exchange-PAINT secondary antibodies coupled to DNA sequences: goat anti-mouse I1, goat anti-chicken I2, goat anti-rabbit I3 and donkey anti-rat I4 for 1.5 hours at RT (see Ab conjugation section for antibody and sequence details). After rinses, they were incubated in phalloidin-Atto488 (Sigma) at 12.5 \textmu{}M for 90 min at RT and imaged within a few days.
For STORM/PAINT imaging, the NanoJ-Fluidics array was installed on an N-STORM microscope (Nikon) equipped with 405, 488, 561 and 647 nm lasers (25, 80, 80 and 125 mW at the optical fiber output). First, a STORM image of phalloidin-ATTO488 was performed in buffer C (PBS 0.1M pH7.2, 500 mM NaCl) using 30,000 frames at 30 ms/frame at 50\% power of the 488 nm laser. After injection of the I1-ATTO655 and I2-CY3B imagers in buffer C, 60,000 frames were aquired in an alternating way (647 nm at 60\% and 561 nm at 30\%) to image TOM20 and vimentin, respectively. After three rinses with buffer C, I3-Cy3B and I4-ATTO655 were in buffer C were injected, and 60,000 frames were aquired in an alternating way (561 nm at 30\% 647 nm at 60\%) to image clathrin and microtubules, respectively. All imager strands were used at concentration between 0.25 and 2 nM. 

\subsection*{Event Detection and Live-to-Fix Imaging}
hTERT-RPE1 cells stably expressing Zyxin-GFP were incubated with 9 mM Ro-3306 (Enzolife Sciences ALX-270-463) to inhibit CDK1 activity for 15-20 hours. Inhibition was released by replacing drug containing media by fresh media at the microscope immediately before imaging. Cells were imaged using a Nikon Eclipse Ti microscope (Nikon) equipped with an Neo-Zyla sCMOS camera (Andor), LED illumination (CoolLED) and a 60X objective (Plan Apo 60X/1.4 Oil, Nikon). Images were acquired every 3 min until enough cells had underwent mitotic rounding. At that point, 16\% warmed PFA was added to cells in media to a final concentration of 4\%, and incubated at room temperature for 20 minutes. They were then washed 3 times and 0.2\% Triton was added for 5 minutes. 5\% BSA in 1X PBS was used to block for 30 min at room temperature, before activated β1 Integrin (Abcam \#ab30394) primary antibody was added. After incubation and washing, Phalloidin-TRITC (Sigma-Aldrich) and anti-mouse AF647 antibody (Invitrogen) were added. All of these steps were performed automatically using the NanoJ-Fluidics platform.

\subsection*{SMLM and SRRF image reconstruction}
For Fig. \ref{fig:LiveToFix} images were reconstructed using NanoJ-SRRF~\cite{gustafsson2016fast} (TRPPM for live cell data and TRM for fixed cell data with a magnification of 4). Drift was estimated using the inbuilt function in NanoJ-SRRF and correction applied during SRRF analysis.\\
For figure \ref{fig:PAINT} localizations were detected using the N-STORM software (Nikon), and exported as a text file before being filtered and rendered using ThunderSTORM \cite{ovesny2014thunderstorm}. Chromatic aberration between the red (561 nm) and far-red (647 nm) channels were corrected within the N-STORM software using polynomial warping, and remaining translational drift between acquisition passes were aligned manually on high-resolution reconstructions.\\
FRC values were obtained using NanoJ-SQUIRREL after reconstruction of original data separated into two two different stacks composed of odd or even images~\cite{culley2018quantitative}. 
NanoJ-SRRF, NanoJ-SQUIRREL and ThunderSTORM are available in Fiji~\cite{schindelin2012fiji}.\\



\section*{Bibliography}
\bibliographystyle{zHenriquesLab-StyleBib}
\bibliography{06_Bibliography_Clean}